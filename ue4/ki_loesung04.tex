\documentclass[11pt,a4paper]{article}
\usepackage{amssymb, amsmath,anysize}
\usepackage{amsmath}
\usepackage[german]{babel}
\usepackage[applemac]{inputenc}
\usepackage{fancyhdr}
\usepackage[T1]{fontenc}
\pagestyle{fancy} \setlength{\parindent}{0cm}
\usepackage{graphicx}
\usepackage{listings}


\begin{document}

\lhead{Samuel\\Lukas}\chead{KI} \rhead{4. Aufgabenblatt} 
\lfoot{} \cfoot{\thepage} \rfoot{}

\bigskip
\bigskip
\bigskip
\bigskip


\section*{1. Aufgabe}
\subsection*{a.}
\begin{equation*}
\begin{split}
\forall Z \exists Y \forall X &(f(X,Y) <=> (f(X,Z) \& \sim f(X,X)))
\end{split}
\end{equation*}

Simplify\\
$\forall Z \exists Y \forall X (f(X,Y) => (f(X,Z) \& \sim f(X,X))) \& ((f(X,Z) \& \sim f(X,X)) => f(X,Y))$ \\
$\forall Z \exists Y \forall X (\sim f(X,Y) | (f(X,Z) \& \sim f(X,X))) \& (\sim (f(X,Z) \& \sim f(X,X)) | f(X,Y))$ \\

Move negations in\\
$\forall Z \exists Y \forall X (\sim f(X,Y) | (f(X,Z) \& \sim f(X,X))) \& (\sim f(X,Z) | f(X,X) | f(X,Y))$ \\

Skolemize\\
$ \exists Y \forall X (\sim f(X,Y) | (f(X,Z) \& \sim f(X,X))) \& (\sim f(X,Z) | f(X,X) | f(X,Y))$ \\
$\gamma = \{Z\}$ \\
$ \forall X (\sim f(X,skY(Z)) | (f(X,Z) \& \sim f(X,X))) \& (\sim f(X,Z) | f(X,X) | f(X,skY(Z)))$ \\
$\gamma = \{Z\}$ \\
$ (\sim f(X,skY(Z)) | (f(X,Z) \& \sim f(X,X))) \& (\sim f(X,Z) | f(X,X) | f(X,skY(Z)))$ \\
$\gamma = \{Z,X\}$ \\

Distribute disjunctions\\
$ (\sim f(X,skY(Z)) | f(X,Z)) \& (\sim f(X,skY(Z)) | \sim f(X,X)) \& (\sim f(X,Z) | f(X,X) | f(X,skY(Z)))$ \\

Convert to CNF\\
$\{\sim f(X,skY(Z)) | f(X,Z) ,  \sim f(X,skY(Z)) | \sim f(X,X) ,  \sim f(X,Z) | f(X,X) | f(X,skY(Z)) \}$ \\

\subsection*{b.}
\begin{equation*}
\begin{split}
\forall X \forall Y &(q(X,Y) <=> \forall Z (f(Z,X) <=> f(Z,Y))) \\
\intertext{Simplify}
\forall X \forall Y &(\sim q(X,Y) | \forall Z ((\sim f(Z,X) | f(Z,Y)) \& (\sim f(Z,Y) | f(Z,X)))) \\
                 \& &(\sim \forall Z ((\sim f(Z,X) | f(Z,Y)) \& (\sim f(Z,Y) | f(Z,X))) | q(X,Y)) \\
\intertext{Move negations in}
\forall X \forall Y &(\sim q(X,Y) | \forall Z ((\sim f(Z,X) | f(Z,Y)) \& (\sim f(Z,Y) | f(Z,X)))) \\
                 \& &(\exists Z ((f(Z,X) \& \sim f(Z,Y)) | (f(Z,Y) \& \sim f(Z,X))) | q(X,Y)) \\
\intertext{Rename variables}
\forall X \forall Y &(\sim q(X,Y) | \forall Z ((\sim f(Z,X) | f(Z,Y)) \& (\sim f(Z,Y) | f(Z,X)))) \\
                 \& &(\exists A ((f(A,X) \& \sim f(A,Y)) | (f(A,Y) \& \sim f(A,X))) | q(X,Y)) \\
\intertext{Skolemize}
                    &(\sim q(X,Y) | ((\sim f(Z,X) | f(Z,Y)) \& (\sim f(Z,Y) | f(Z,X)))) \\
                 \& &(((f(skA(X,Y),X) \& \sim f(skA(X,Y),Y)) | (f(skA(X,Y),Y) \& \sim f(skA(X,Y),X))) | q(X,Y)) \\
\intertext{Distribute disjunctions}
                    &(\sim q(X,Y) | \sim f(Z,X) | f(Z,Y)) \\
                 \& &(\sim q(X,Y) | \sim f(Z,Y) | f(Z,X)) \\
                 \& &(f(skA(X,Y),X) | f(skA(X,Y),Y) | q(X,Y)) \\
                 \& &(f(skA(X,Y),X) | \sim f(skA(X,Y),X) | q(X,Y)) \\
                 \& &(\sim f(skA(X,Y),Y) | f(skA(X,Y),Y) | q(X,Y)) \\
                 \& &(\sim f(skA(X,Y),Y) | \sim f(skA(X,Y),X) | q(X,Y)) \\
\intertext{Convert to CNF}
                 \{ &(\sim q(X,Y) | \sim f(Z,X) | f(Z,Y)) \\
                 ,  &(\sim q(X,Y) | \sim f(Z,Y) | f(Z,X)) \\
                 ,  &(f(skA(X,Y),X) | f(skA(X,Y),Y) | q(X,Y)) \\
                 ,  &(f(skA(X,Y),X) | \sim f(skA(X,Y),X) | q(X,Y)) \\
                 ,  &(\sim f(skA(X,Y),Y) | f(skA(X,Y),Y) | q(X,Y)) \\
                 ,  &(\sim f(skA(X,Y),Y) | \sim f(skA(X,Y),X) | q(X,Y)) \} \\
\end{split}
\end{equation*}


\section*{3. Aufgabe}
Folgendes Herbrand Universum und folgende Herbrand Basis besitzt das Modell.
\begin{equation*}
\begin{split}
HU = \{ & max, \\
        & vater\_von(max), \\
        & mutter\_von(max), \\
        & vater\_von(vater\_von(max)), \\
        & vater\_von(mutter\_von(max)), \\
        & mutter\_von(vater\_von(max)),\\
        & mutter\_von(mutter\_von(max)),\\
        & vater\_von(vater\_von(vater\_von(max))),\\
        & vater\_von(vater\_von(mutter\_von(max))),\\
        & vater\_von(mutter\_von(vater\_von(max))),\\
        & vater\_von(mutter\_von(mutter\_von(max))),\\
        & ... \}
\end{split}
\end{equation*}
\begin{equation*}
\begin{split}
HB = \{ & verheiratet(max,max),\\
        & verheiratet(max,vater\_von(max)),\\
        & verheiratet(vater\_von(max),max),\\
        & verheiratet(vater\_von(max),vater\_von(max)),\\
        & verheiratet(max,mutter\_von(max)),\\
        & verheiratet(vater\_von(max),mutter\_von(max)),\\
        & verheiratet(mutter\_von(max),mutter\_von(max)),\\
        & verheiratet(mutter\_von(max),max),\\
        & verheiratet(mutter\_von(max),vater\_von(max)),\\
        & verheiratet(max,vater\_von(vater\_von(max))),\\
        & ... \}
\end{split}
\end{equation*}
Folgende Herbrand Interpretation mit $verheiratet(vater\_von(max),mutter\_von(max))$ ist m\"oglich:
\begin{equation*}
\begin{split}
D = \{ & \ddot\smile, \\
       & v(\ddot\smile), \\
       & m(\ddot\smile), \\
       & v(v(\ddot\smile)), \\
       & v(m(\ddot\smile)), \\
       & ... \}
\end{split}
\end{equation*}
\begin{equation*}
\begin{split}
F  = \{ & max \rightarrow \ddot\smile,\\
        & vater\_von(max) \rightarrow v(\ddot\smile),\\
        & mutter\_von(max)\rightarrow  m(\ddot\smile),\\
        & vater\_von(vater\_von(max))\rightarrow v(v(\ddot\smile)), \\
        & vater\_von(mutter\_von(max)) \rightarrow v(m(\ddot\smile)), \\
        & ... \}
\end{split}
\end{equation*}
\begin{equation*}
\begin{split}
R  = \{ & verheiratet(\ddot\smile,\ddot\smile) \rightarrow FALSE,\\
        & verheiratet(\ddot\smile,v(\ddot\smile)) \rightarrow FALSE,\\
        & verheiratet(v(\ddot\smile),\ddot\smile) \rightarrow FALSE,\\
        & verheiratet(v(\ddot\smile),v(\ddot\smile)) \rightarrow FALSE,\\
        & verheiratet(\ddot\smile,m(\ddot\smile)) \rightarrow FALSE,\\
        & verheiratet(v(\ddot\smile),m(\ddot\smile)) \rightarrow TRUE,\\
        & verheiratet(m(\ddot\smile),m(\ddot\smile)) \rightarrow FALSE,\\
        & verheiratet(m(\ddot\smile),\ddot\smile) \rightarrow FALSE,\\
        & verheiratet(m(\ddot\smile),v(\ddot\smile)) \rightarrow TRUE,\\
        & verheiratet(\ddot\smile,v(v(\ddot\smile))) \rightarrow FALSE,\\
        & ... \}
\end{split}
\end{equation*}

\end{document}
