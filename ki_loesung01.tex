
\documentclass{article}

\begin{document}
\section*{2. Aufgabe}{
Beim Chinese Room Thought Experiment geht es darum, dass auch menschliche Intelligenz k\"unstlich werden k\"onnte.

Ein absolut Sprachunkundiger soll sich mithilfe eines Regelwerkes verst\"andigen, ohne n\"aheres Erfahren der Bedeutung des Gesagten. Die Gespr\"achspartner sollen jedoch vermuten, dass der Sprachunkundige w\"usste, wovon er erz\"ahlt.

Diese Situation \"ahnelt dem Turing-Test und soll eine Art Argument daf\"ur liefern, warum Strong-AI immer noch nicht intelligent sei.

In einer anderen Sichtweise l\"asst sich das Gegenargument bringen, dass sobald der Sprachunkundige sich authentisch verst\"andigt, ein Teil von ihm bzw. sein Regelwerk das Gesagte durchaus verstehen muss, auch wenn er sich dessen nicht bewusst ist. Meiner Meinung nach ist es kein Gegenargument daf\"ur, k\"unstliche Intelligenz w\"are nicht Intelligent.
}
\end{document}
